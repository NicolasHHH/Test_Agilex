\documentclass{article}

\usepackage{amsmath}

\begin{document}

    \title{Linear Quadratic Regulator}
    \date{\today}

    \maketitle

    \section{Introduction}

    A Linear Quadratic Regulator (LQR) is an optimal control algorithm used in control theory. The method involves calculating a control signal that minimizes a cost function, typically composed of state error and control effort.

    \section{Algorithm Steps}

    \begin{enumerate}

        \item \textbf{System Definition}

        We start with a linear system of the form:
        \begin{align*}
            \dot{x} &= Ax + Bu \\
            y &= Cx + Du
        \end{align*}
        where $x$ is the state vector, $u$ is the control input vector, $A$ and $B$ are system matrices, and $y$ is the output.

        \item \textbf{Cost Function Definition}

        The cost function to minimize is:
        \[
            J = \int_{0}^{\infty} (x^TQx + u^TRu) \, dt
        \]
        where $Q$ and $R$ are weight matrices that decide the importance of state error and control effort.

        \item \textbf{Solving the Riccati Equation}

        The solution to the LQR problem is given by the Riccati equation:
        \[
            A^TPA - P - A^TPB(B^TPB + R)^{-1}B^TPA + Q = 0
        \]

        \item \textbf{Iterative Solution}

        Iteratively solve the Riccati equation until $P$ converges within some small tolerance $\varepsilon$. Then compute the feedback gain:
        \[
            K = (B^TPB + R)^{-1}B^TPA
        \]

        \item \textbf{Control Signal Calculation}

        Calculate the control signal as:
        \[
            u = -Kx
        \]

        \item \textbf{System Control}

        Apply the control signal $u$ to the system. The control is designed to minimize the cost function while driving the system towards the desired state.

    \end{enumerate}

\end{document}
